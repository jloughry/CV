% The cmap package avoids the problem of ligatures like 'ffi' disappearing from a
% copy and paste operation where the destination is plain text. This is important
% when sending out résumés in PDF if you want keywords to be spotted by automated
% filters used by HR. Without the `resetfonts' option it doesn't do the trick with
% Computer Modern Roman.
%
% I use a modified ot1.cmap and a modified oms.cmap and a modified cmap.sty file
% in the current directory to tell PDF that in addition to the ligatures mentioned
% above it should also replace curly quotes with straight quotes, en dashes and em
% dashes with hyphens, and bullets with asterisks or other ASCII characters. Copy
% and paste of text from the resulting PDF file will be 7-bit clean ASCII containing
% no Unicode characters at all. Be sure to load cmap immediately after the
% documentclass line so it can see all the fonts as they are loaded. This version
% of cmap.sty works with 11pt and 12pt options to documentclass unlike the original.
\usepackage[resetfonts]{cmap}

% The geometry package lets me control the page margins.
\usepackage[margin=0.75in,top=1in]{geometry}

% The titling package lets me control white space at the top of the first page.
\usepackage{titling}
\setlength{\droptitle}{-37mm}

% Used for graphics in the social networking bullets
\usepackage{graphicx}

% Used to typeset a plus sign that looks right in text mode, not math mode. Source:
% \url{http://tex.stackexchange.com/questions/52503/sign-in-international-phone-numbers}.
\providecommand{\plus}{\raisebox{.4\height}{\scalebox{.6}{+}}}

% Modify the trademark symbol so it will paste to plain text the way I want it to.
\providecommand{\rmtrademark}{\raisebox{\height}{\tiny (TM)}}

% The xcolor package is used to grey out text in the Social Networking section at
% the end (although it doesn't grey out the graphic icons on those lines).
\usepackage{xcolor}

% The url package formats URLs and works with hyperref (which must be loaded last).
\usepackage{url}

% The hyperref package makes links show up in generated PDF files.
\usepackage{hyperref}

% Fill in some information in the PDF header. Note: if the pdfkeywords line is split,
% then it will end up with quotation marks placed around it. No commas allowed.

\hypersetup{pdfauthor=Joe Loughry}
\hypersetup{pdftitle=R\'{e}sum\'{e}}
\hypersetup{pdfsubject=computer security}
\hypersetup{pdfkeywords=Software development and Certification and Accreditation (C\&A) of Cross Domain Systems (CDS) and project lead}

% Set it to print two-sided, if the printer supports it (user can override this setting)
\hypersetup{pdfduplex=DuplexFlipLongEdge}

% For commenting out sections:
\usepackage{comment}

\textheight=9.5in
\raggedbottom

% The following trick changes the margins in a \quote environment. Source: \url{http://tex\
% .stackexchange.com/questions/2396/how-can-i-change-the-indentation-in-quote-and-quotation-\
% environments-and-command}.
\newenvironment{myquote}{\list{}{\leftmargin=0.15in\rightmargin=0in}\item[]}{\endlist}

% Tell \LaTeX\ not to complain about missing date.
\def\nodate{\@makedatefalse}
\date{}

\title{\textsc{Joe Loughry}}
\author{}

