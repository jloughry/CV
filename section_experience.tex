\section*{}

\begin{comment}
\vspace{-10mm}
\emph{Technical programming, cybersecurity, certification and accreditation
of cross domain solutions and systems, cryptography, pentesting, security
research, writing and speaking. Specialist in understanding new government
computer security standards---especially when they're new or have
suddenly changed. Experienced in Common Criteria, DIACAP, NIST SP 800-53
and RMF, DCID 6/3, and DoD 8540.}
\end{comment}

\vspace{-10mm}
\begin{comment}
\noindent\textbf{Research Assistant Professor}
    \hfill\emph{University of Denver}\hfill 2017--present

    \vspace{1mm}
    \noindent Developing a transformational new method for high assurance cross-domain
    information transfer. FPGA implementation from a
    security-approval--forward basis. The intended application is
    between multiple international partners and intel community.
    Principal Investigator (PI) for the CCM project.

\vspace{3mm}
\end{comment}
\noindent\textbf{Researcher}
    \hfill\emph{C\&Adocs, Inc.} \hfill 2017--present

    \vspace{1mm}
    \noindent 
    Currently working on a route to low cost high assurance cross-domain systems
    inherently immune to malware. FPGA or VLSI implementation from a
    security-approval--forward basis. The intended application is between
    multiple international partners and the intelligence community.

\vspace{3mm}
\noindent\textbf{Visiting Assistant Professor}
    \hfill \emph{University of Denver} \hfill 2016--2017

    \vspace{1mm}
    \noindent Taught computer security, beginning C programming, and ethical hacking
    classes. Lecture and laboratory demonstration in cyberweapons,
    counterintelligence, TEMPEST countermeasures. Presented a special public lecture on
    current topics in cybersecurity, Snowden, and the U.S.\ government.

\vspace{3mm}
\noindent\textbf{Consultant}
    \hfill \emph{C\&Adocs, Inc.} \hfill 2015--2016

    \vspace{1mm}
    \noindent Secret-and-Below Interoperability (SABI) certification
    testing of two new cross domain solutions (CDS) for a developer in the
    D.C.\ area. Lots of experience with NIST 800-53 security controls:
    selection and arguing them with the certifier. Advising on likely certifier
    directions and anticipating certifier moves. Writing documentation to
    continually evolving requirements. Vulnerability analysis.

\vspace{2mm}
\noindent\textbf{Postgraduate Researcher}
    \hfill \emph{University of Oxford} \hfill 2007--2015

    \vspace{1mm}
    \noindent Discovered methods to control the schedule and predict the
    outcome of security Certification and Accreditation (C\&A)
    testing of cross domain solutions and cross domain systems for
    Intelligence Community (IC), collateral, and international
    environments, with applications in the area of privacy protection
    of Electronic Health Record (EHR) systems.

\vspace{2mm}
\noindent\textbf{Information Assurance Engineer}
    \hfill \emph{Lockheed Martin IS\&GS} \hfill 2006--2012

    \vspace{1mm}
    \noindent Proposed and won \$968,000 Air Force Research Laboratory
    contract for probabilistic redaction system; R\&D
    project completed on time and under budget. Wrote the
    Security Target for Common Criteria (CC) evaluation of Radiant
    Mercury\rmtrademark. Built comparison tool for common international
    security accreditation standards for F-35 Joint Strike Fighter.
    Primary interface with UK's GCHQ.

\vspace{2mm}
\noindent\textbf{Senior Software Engineer}
    \hfill \emph{Lockheed Martin Missiles \& Space Company} \hfill 1998--2006

    \vspace{1mm}
    \noindent Invented nested digital signatures for satellite imagery
    files---U.S.\ patent no.~8,793,499 issued. Discovered the
    optical TEMPEST effect and its countermeasures---U.S.\ patent
    no.~6,987,461. Software developer and security engineer with
    responsibility for the real-time performance of Radiant
    Mercury\rmtrademark\ on the B-2 stealth bomber.

\begin{comment}
\vspace{2mm}
\noindent\textbullet\ \textbf{University Instructor, CSSE 591 (Computer Networks)}
    \hfill \emph{Seattle University} \hfill Spring 1998

    \vspace{1mm}
    \noindent Taught practical networking design from a business perspective, layers
    1 through 4 and TCP/IP. NICs, bridges, routers, and firewalls.
    IP numbering, DHCP and DNS, file and print servers. Network
    backbone design trade-offs, structured cabling, security \& data
    encryption, storage and mail servers.
\end{comment}
